%%%%%%%%%%%%%%%%%%%%%%%%%%%%%%%%%%%%%%%%%
% Wilson Resume/CV
% XeLaTeX Template
% Version 1.0 (22/1/2015)
%
% This template has been downloaded from:
% http://www.LaTeXTemplates.com
%
% Original author:
% Howard Wilson (https://github.com/watsonbox/cv_template_2004) with
% extensive modifications by Vel (vel@latextemplates.com)
%
% License:
% CC BY-NC-SA 3.0 (http://creativecommons.org/licenses/by-nc-sa/3.0/)
%
%%%%%%%%%%%%%%%%%%%%%%%%%%%%%%%%%%%%%%%%%

%----------------------------------------------------------------------------------------
%	PACKAGES AND OTHER DOCUMENT CONFIGURATIONS
%----------------------------------------------------------------------------------------

\documentclass[10pt]{article} % Default font size



\usepackage{tabularx}
% \usepackage[margin=0.8in]{geometry}
\addtolength{\oddsidemargin}{-1.0in}
\addtolength{\evensidemargin}{-1.0in}
\addtolength{\textwidth}{2.0in}

\addtolength{\topmargin}{-1.0in}
\addtolength{\textheight}{6.0in}

\usepackage{fontspec}
\usepackage{xcolor}
\usepackage{array}
\usepackage{tikz}
\usetikzlibrary{calc}
\usepackage{zref-savepos}
\usepackage[none]{hyphenat}

\newcolumntype{b}{X}
% \newcolumntype{s}{>{\hsize=.385\hsize\raggedright\let\newline\\\arraybackslash\hspace{0pt}
\newcolumntype{s}{>{\hsize=.385\hsize\raggedright
                    % \let\newline\\
                    \arraybackslash
                    % \hspace{0pt}
}X}
% \newcolumntype{s}{>{\hsize=.385 \hsize}X}
\newcolumntype{m}{>{\hsize=.015\hsize}X}
\setmainfont{Liberation Sans}
% \setmainfont{Arial}
%----------------------------------------------------------------------------------------
\addtolength{\parskip}{-0.5mm}

\definecolor{resume_green}{HTML}{3FBB69}

\begin{document}

\pagenumbering{gobble}

\newcounter{NoTableEntry}
\renewcommand*{\theNoTableEntry}{NTE-\the\value{NoTableEntry}}

\newcommand{\tikzmark}[1]{\tikz[overlay,remember picture] \node (#1) {};}
\newcommand{\DrawLine}[3][]{%
    \begin{tikzpicture}[overlay,remember picture]
        \draw [#1] ($(#2)+(0,0.6ex)$) -- ($(#3)+(0,0.6ex)$);
    \end{tikzpicture}%
}%
\newcommand{\resumeRow}[2]{%
    \begin{tabularx}{\textwidth}{smb}
        #1
        &~&
        #2
    \end{tabularx}
}%

\newcommand{\divider}[1]{%
    ~\newline
    \resumeRow{
        \tikzmark{StartA}\hspace*{\fill}\tikzmark{EndA}
    }{
        \textcolor{resume_green}{\large #1}
            % \tikz[baseline=-0.5ex]\draw[thick, dotted] (1,0) -- (5,0);
        \hspace{3em}\tikzmark{StartB} \hspace*{\fill}\tikzmark{EndB}
    }
    \DrawLine[thick, dotted]{StartA}{EndA}
    \DrawLine[thick, dotted]{StartB}{EndB}
}%
%%%%%%%%%%%%%%%%%%%%%%%%%%%%%%%%%%%%%%%%%%%%%%%%%%%%%%%%%%%%%%%%%%%%%%
%%%%%%%%%%%%%%%%%      END PREAMBLE       %%%%%%%%%%%%%%%%%%%%%%%%%%%%
%%%%%%%%%%%%%%%%%%%%%%%%%%%%%%%%%%%%%%%%%%%%%%%%%%%%%%%%%%%%%%%%%%%%%%

% \title{Zane Dufour - Software Engineer} % Print the main header
~% For some reason, this tilde is important (for keeping everything vertically justified)
% \maketitle
\begin{table}
    \center
    \resumeRow{
        \begin{Huge}
            ~\newline
        \end{Huge}
        \textcolor{resume_green}{MOBILE} \newline
        +1 (310) 600-8638
        \newline\newline
        \textcolor{resume_green}{EMAIL} \newline
        zanedufour@berkeley.edu 
    }{
        \textcolor{resume_green}{\Huge\bf Zane Dufour}   
        \newline
        \newline
        I intend to pursue junior-level software/data engineering positions. 
    }
    \\
    %&&&&&&&&&&&&&&%&&&&&&&&&&&&&&%&&&&&&&&&&&&&&
    %&&&&&&&&&&&&&&  EDUCATION  %&&&&&&&&&&&&&&%&
    %&&&&&&&&&&&&&&%&&&&&&&&&&&&&&%&&&&&&&&&&&&&&
    \divider{EDUCATION}
    \\
    \resumeRow{
        UC Berkeley, \newline
        May 2017
    }{
        Double Bachelor's -- Applied Math and Physics\newline
        GPA 3.4
    }
    \\
    \divider{EXPERIENCE}
    \\
    \resumeRow{
        \textbf{Ford Motor Company} \newline
        Analytics Developer \newline
        Dearborn, MI \newline
        November 2017 - Present
    }{
        While on a small modeling team, I developed likelihood-to-purchase models on hundreds of millions of individuals. I refactored the team's modeling pipeline -- I helped the team adopt Github for version control -- and moved data transformation code from Jupyter notebooks into abstracted functions. Created a python package that simplified connecting to Ford's pyspark distributed computing cluster. Utilized test-driven-development practices while building out a suite of tools to simplify machine learning workflows, including a web application and multiple python libraries. Set up dependency management and automated testing using conda build and tox. While working on the customer analytics team, used plotly to create interactive data visualizations for analytics reports.
    }
    \\\vspace{.3cm}
    \resumeRow{
        \textbf{Disney Imagineering} \newline
        Software Engineering Intern \newline
        Glendale, CA \newline
        June-September 2017
    }{
        While working in the Disney Imagineering Media and Art Pipeline group, I developed software used for projection mapping in Disney parks and resorts. I built a continuous integration system for multiple interdependent applications which were used for diferent parts of the projection mapping pipeline.
    }
    \\\vspace{.3cm}
    \resumeRow{
        \textbf{Intel}\newline
        Software Engineering Intern\newline
        Santa Clara, CA\newline
        February-August 2016
    }{
        During this six month internship at Intel, I developed manufacturing and design tools for the Silicon Photonics group. While on this team, I added an exception-handler and a sqlite logging system. This was the first time I maintained a large code base and learned about writing reusable code.
    }
    \\\vspace{.3cm}
    \resumeRow{
        \textbf{UC Berkeley}\newline
        Research Assistant\newline
        Computational Geometry\newline
        Summer 2015 - Fall 2016
    }{
        While working as an undergraduate research assistant, I worked on a
        spectral geometry morpher in C++ and a Houdini tool for generating parameterized geometry.
    }
    %%%%%%%%%%%%%%%%%%%%%%%%%%%%%%%%%%%%%%%%%%%%%%%%%%
    %%%%%%%%%%%%%%%%   COURSEWORK    %%%%%%%%%%%%%%%%%
    %%%%%%%%%%%%%%%%%%%%%%%%%%%%%%%%%%%%%%%%%%%%%%%%%%
    \\
    \divider{COURSES}
    \\
    \resumeRow{
        \textbf{Machine Learning}
    }{
        Built various machine learning models from scratch in Python w/ NumPy. This included Character Recognition models, SVMs, Neural Networks, Gausian Discriminant Analysis, Decision Trees and Random Forests.
    }
    \\\vspace{.3cm}
    \resumeRow{
        \textbf{Spectral Methods in Computational Fluid Dynamics (Graduate)}
    }{
        Used NumPy to find numerical solutions to Poisson and Navier-Stokes Equations. Implemented Runge-Kutta finite step methods, Fast Fourier and Chebyshev transforms. 
    }
    \\\vspace{.3cm}
    \resumeRow{
        \textbf{Advanced Linear Algebra}
    }{
        Diagonalizing Matrices; Isomorphic Vector Spaces; Inner product
        spaces; change of basis; Singular Value Decompositions
    }
    \\\vspace{.3cm}
    \resumeRow{
        \textbf{Analytical Mechanics}
    }{
        Lagrangian and Hamiltonian Mechanics; Orbital Mechanics;
        Chaos and Instability; Rigid-Body kinematics
    }

\end{table}
% ~

%-------------------------------------------------------------------------

\end{document}