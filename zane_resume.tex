%%%%%%%%%%%%%%%%%%%%%%%%%%%%%%%%%%%%%%%%%
% Wilson Resume/CV
% XeLaTeX Template
% Version 1.0 (22/1/2015)
%
% This template has been downloaded from:
% http://www.LaTeXTemplates.com
%
% Original author:
% Howard Wilson (https://github.com/watsonbox/cv_template_2004) with
% extensive modifications by Vel (vel@latextemplates.com)
%
% License:
% CC BY-NC-SA 3.0 (http://creativecommons.org/licenses/by-nc-sa/3.0/)
%
%%%%%%%%%%%%%%%%%%%%%%%%%%%%%%%%%%%%%%%%%

%----------------------------------------------------------------------------------------
%	PACKAGES AND OTHER DOCUMENT CONFIGURATIONS
%----------------------------------------------------------------------------------------

\documentclass[10pt]{article} % Default font size



\usepackage{tabularx}
% \usepackage[margin=0.8in]{geometry}
\addtolength{\oddsidemargin}{-.875in}
\addtolength{\evensidemargin}{-.875in}
\addtolength{\textwidth}{1.75in}

\addtolength{\topmargin}{-1.0in}
\addtolength{\textheight}{4.0in}

\usepackage{fontspec}
\usepackage{xcolor}
\usepackage{array}
\usepackage{tikz}
\usetikzlibrary{calc}
\usepackage{zref-savepos}
\usepackage[none]{hyphenat}

\newcolumntype{b}{X}
% \newcolumntype{s}{>{\hsize=.385\hsize\raggedright\let\newline\\\arraybackslash\hspace{0pt}
\newcolumntype{s}{>{\hsize=.385\hsize\raggedright
                    % \let\newline\\
                    \arraybackslash
                    % \hspace{0pt}
}X}
% \newcolumntype{s}{>{\hsize=.385 \hsize}X}
\newcolumntype{m}{>{\hsize=.015\hsize}X}
% \setmainfont{Liberation Sans}
\setmainfont{Arial}
%----------------------------------------------------------------------------------------
\addtolength{\parskip}{-0.5mm}

\definecolor{resume_green}{HTML}{3FBB69}

\begin{document}

\pagenumbering{gobble}

\newcounter{NoTableEntry}
\renewcommand*{\theNoTableEntry}{NTE-\the\value{NoTableEntry}}

\newcommand{\tikzmark}[1]{\tikz[overlay,remember picture] \node (#1) {};}
\newcommand{\DrawLine}[3][]{%
    \begin{tikzpicture}[overlay,remember picture]
        \draw [#1] ($(#2)+(0,0.6ex)$) -- ($(#3)+(0,0.6ex)$);
    \end{tikzpicture}%
}%
\newcommand{\resumeRow}[2]{%
    \begin{tabularx}{\textwidth}{smb}
        #1
        &~&
        #2
    \end{tabularx}
}%

\newcommand{\divider}[1]{%
    ~\newline
    \resumeRow{
        \tikzmark{StartA}\hspace*{\fill}\tikzmark{EndA}
    }{
        \textcolor{resume_green}{\large #1}
            % \tikz[baseline=-0.5ex]\draw[thick, dotted] (1,0) -- (5,0);
        \hspace{3em}\tikzmark{StartB} \hspace*{\fill}\tikzmark{EndB}
    }
    \DrawLine[thick, dotted]{StartA}{EndA}
    \DrawLine[thick, dotted]{StartB}{EndB}
}%
%%%%%%%%%%%%%%%%%%%%%%%%%%%%%%%%%%%%%%%%%%%%%%%%%%%%%%%%%%%%%%%%%%%%%%
%%%%%%%%%%%%%%%%%      END PREAMBLE       %%%%%%%%%%%%%%%%%%%%%%%%%%%%
%%%%%%%%%%%%%%%%%%%%%%%%%%%%%%%%%%%%%%%%%%%%%%%%%%%%%%%%%%%%%%%%%%%%%%

% \title{Zane Dufour - Software Engineer} % Print the main header
~% For some reason, this tilde is important (for keeping everything vertically justified)
% \maketitle
\begin{table}
    \center
    \resumeRow{
        \begin{Huge}
            ~\newline
        \end{Huge}
        \textcolor{resume_green}{MOBILE} \newline
        +1 (310) 600-8638
        \newline\newline
        \textcolor{resume_green}{EMAIL} \newline
        zanedufour@berkeley.edu 
    }{
        \textcolor{resume_green}{\Huge\bf Zane Dufour}   
        \newline
        \newline
        I intend to pursue a junior-level software engineering position. 
    }
    \\
    %&&&&&&&&&&&&&&%&&&&&&&&&&&&&&%&&&&&&&&&&&&&&
    %&&&&&&&&&&&&&&  EDUCATION  %&&&&&&&&&&&&&&%&
    %&&&&&&&&&&&&&&%&&&&&&&&&&&&&&%&&&&&&&&&&&&&&
    \divider{EDUCATION}
    \\
    \resumeRow{
        UC Berkeley, \newline
        May 2017
    }{
        Double Bachelor's -- Applied Math and Physics\newline
        GPA 3.4
    }
    \\
    \divider{EXPERIENCE}
    \\
    \resumeRow{
        Ford Motor Company \newline
        Analytics Developer \newline
        Dearborn, MI \newline
        November 2017 - 
    }{
        While working at Ford, I worked to automate steps in my team's 
        model creation process to accelerate the creation of likelihood-to-purchase models. Since then, I have been working on a concept-to-production model development pipeline utilizing Python and Spark to build and deploy models at scale.
    }
    \\\vspace{.3cm}
    \resumeRow{
        Disney Imagineering \newline
        Software Imagineer \newline
        Glendale, CA \newline
        June-September 2017
    }{
        While at Disney, I developed software used for projection mapping in Disney parks and resorts. I built a continuous integration system for multiple interdependent applications used for diferent parts of the projection mapping pipeline.
    }
    \\\vspace{.3cm}
    \resumeRow{
        Intel Corporation\newline
        Design Automation Intern\newline
        Santa Clara, CA\newline
        February-August 2016
    }{
        During this six month internship at Intel, I developed manufacturing and design tools for the Silicon Photonics group. I built an Exception-handler and a local database client. During this internship I learned about maintaining a large code base and writing reusable code.
    }
    \\\vspace{.3cm}
    \resumeRow{
        UC Berkeley\newline
        Research Assistant\newline
        Computational Geometry\newline
        Summer 2015 - Fall 2016
    }{
        Scripted a polygonal geometry morpher in the software package Houdini while working under the instruction of Professor Philip Marcus. Was most recently working on surface parameterization.
    }
    %%%%%%%%%%%%%%%%%%%%%%%%%%%%%%%%%%%%%%%%%%%%%%%%%%
    %%%%%%%%%%%%%%%%   COURSEWORK    %%%%%%%%%%%%%%%%%
    %%%%%%%%%%%%%%%%%%%%%%%%%%%%%%%%%%%%%%%%%%%%%%%%%%
    \\
    \divider{COURSES}
    \\
    \resumeRow{
        Machine Learning
    }{
        Building ML algorithms from scratch in Python w/ NumPy.
        Character Recognition, SVMs, Neural Networks,
        Gausian Discriminant Analysis, Decision Trees etc.
    }
    \\\vspace{.3cm}
    \resumeRow{
        Spectral Methods in Computational Fluid Dynamics (Graduate)
    }{
        Using NumPy to find numerical solutions to Poisson and Navier-Stokes
        Equations; Runge-Kutta finite step methods; Fast Fourier and Chebyshev transforms
    }
    \\\vspace{.3cm}
    \resumeRow{
        Advanced Linear Algebra
    }{
        Diagonalizing Matrices; Isomorphic Vector Spaces; Inner product
        spaces; change of basis; Singular Value Decompositions
    }

\end{table}
~

%-------------------------------------------------------------------------

\end{document}